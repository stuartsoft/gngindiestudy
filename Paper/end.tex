\section{Summary}
	Overall, there were many hurtles and challenges to overcome in our genetic algorithm implementation, and there is still much more that could be done to further improve that system that is in place. Further work could still be done to improve the overall runtime of the algorithm, including parallelizing the evaluation and crossover stages. Additionally, the mutation stage could also be improved, by moving pre-existing nodes, or adding and deleting edges. Several constraints could also use tuning, including the number of graphs required in each generation or the percentage of nodes cycled out during each generation. Finally it would help tremendously to run this algorithm with larger mazes and see how it operates in a larger environment where obtaining a high A* satisfaction requires even more thorough maze coverage.
	
	The algorithm in its current state does function as a strong proof of concept for the benefits of genetic algorithms in waypoint graphs. The system was able to achieve a reasonably similar A* satisfaction rating to that of a completely uninformed system, while using substantially fewer nodes in the process. With some additional tuning, this system could serve as a very useful, optimized, fully automated environment waypoint mapping algorithm.


\section{Citations}

\begin{enumerate}
	\item Jacobson L. (2012, Feb 12). “Creating a Genetic Algorithm for Beginners” [Blog]. Available: http://www.theprojectspot.com/tutorial-post/creating-a-genetic-algorithm-for-beginners/3 
	\item McCall, J.“Genetic Algorithms for Modelling and Optimization” Journal of Comp. and Applied Math, vol. 184, no. 1, pp. 205-222, Dec 2005
\end{enumerate}